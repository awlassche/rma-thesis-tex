\chapter{Conclusion and discussion}

For years, literary researchers on the early modern period have published studies on countless well-known and less well-known early modern texts and authors, enough to fill complete libraries. These qualitative studies have shaped our idea about the Dutch Republic, her inhabitants, leaders, enemies, and in particular, her literature. To some extent, it is understandable that these researchers are a bit wary of new, digital methods to investigate cultural products from a larger distance. Zooming out means that many details become invisible: details which, with close reading, would have given us essential information about a text, its writer, publisher, reader, or the context in which it was created. However, in words of Underwood, \enquote{[a] single pair of eyes at ground level can't grasp the curve of the horizon}.\autocite[x]{underwood_distant_2019} If we take our distance and get up in the sky, we will see things, patterns, we were never able to see before. This does not mean that we should forget the observations we did earlier, when we were still on our knees, studying a text through a magnifying glass. The golden mean shall always be a combination of both: a confrontation of the one method with the other, to see how both perspectives are related to each other and how they can complement each other.

That is what I have tried to do in this thesis. In researching a corpus of Dutch early modern songs from a larger distance, I have not only added a study to the very long list of studies on Dutch early modern song culture, but most of all, I have expanded the very short list of \textit{quantitative} studies on the Dutch early modern song. The answers I have found on the twofold question I asked in the first chapter of this thesis, are probably not the most surprising. However, the application of digital methods to a corpus of 43,772 Dutch songs was a rather unexplored area. With the results obtained in this research and the interpretation of these results, I have combined the two perspectives mentioned above (close and distant reading), and evaluated if and how the outcome of qualitative research on the one hand and quantitative research on the other hand can be related to each other. I do believe that the combination of these two methods and the confrontation with each other can help the early modern literary studies reach a new level.

The first of the two questions I asked was which topics the most dominant were in early modern Dutch songs. In answering this question, the preliminary steps I had to take before I was actually able to let the computer compose topics, took the most time. In the previous months, during my internship at the Meertens Institute, I had gained enough knowledge on skills such as parsing XML-files, that this was no longer a problem. However, the enormous variation in spelling of these songs turned out to be a big bump. Working with a corpus containing texts from a period of two hundred years in which no normalized spelling was set, was in this case a very complicating fact. I use the words \enquote{in this case}, because I realize that there are also situations in which this variation is no problem, or even desirable, for example to investigate how fast or slow the spelling of words change over time. In the context of this thesis, however, it was very important that all texts were normalized to the same spelling. For topic modeling it is necessary that a word is spelled exactly the same way through time, otherwise the computer will build different topics for situations in which one topic would be sufficient.

To normalize historical spelling, there are several methods available for the English language, but regarding the Dutch language, options were limited. The very straightforward INL-tool I used had been helpful in a sentiment analysis of early modern songs, but in this study, the tool worked too sloppy and the results were therefore too unreliable. The VARD2-tool I tested had much more potential, especially because the tool gave an output file in which all made changes were reported. The fact that the training set of the tool was build from seventeenth century translations of the Bible, had both advantages and disadvantages. I already knew that a major part of my song corpus consisted of religious songs, so there would be a lot of overlapping words between the Bible and my corpus. The dating of the training set in the seventeenth century, had the disadvantage that older words (from the fifteenth or sixteenth century) were not included in the set. \enquote{Old} words in songs, such as \enquote{dijn}, were therefore not normalized by the tool. Still, I considered the VARD-normalized version of the corpus as the best version, and therefore continued my analysis with that one.

Of all the digital methods to research literary texts computationally that have arisen in the last decades, I have chosen a rather controversial one in this thesis. Topic modeling is widely discussed, not only because it is based on probabilities and the mathematical idea therefore seems a mystery to many users, but also because we know that the tool needs improvement and is not already at its best at the moment. Changing some settings, for example the number of topics or the number of iterations, can highly influence the output. I therefore have had rerun the tool several times, searching for the best settings. In this study, as well as for all studies in which computational methods are used, it is of great importance to critically evaluate the methods used. Although I have done that to some extent, I realize that there are still aspects that could have been evaluated more. Another point where human intervention may have influenced the results negatively, is the assigning of a certain subject to topics. I have done this assigning by looking at the word cloud of a topic, in which more dominant words were depicted larger than less dominant words. I have tried to include this difference in word dominance in my consideration which subject would describe the topic at best, but still it is possible that, if someone else would have done the topic assigning, different subjects would be given to the topics, simply because a human being can never be as objective as a computer -- which, actually, is why a computer is never 100\% objective as well, since I am (or somebody else is) the one building the tool.

This finally brings me to the answer on the first part of the twofold research question, what the most dominant topics in my corpus were. As said before, the answer did not come as a surprise. Based on the genre distribution in my corpus, I already expected the religious topics to be by far the most dominant. However, what the genre distribution of the Dutch Song Database does not show, are the subcategories within a genre. Furthermore, a genre does not say anything about the topic of songs within that genre. Those insights are obtained by building topics from the corpus. Apart from all criticism, it is admirable that the topic modeling tool was able to distinguish between songs on, for example, \textit{love and tragedy}, and songs on \textit{rejection}. Another remarkable observation was the dominance of songs with \textit{nation and country} as their most dominant topic, especially after adding the reprints and songs from other songbooks to the corpus. It is necessary to discuss the method I used in choosing the most dominant topic for each song. The output of a topic modeling tool is always a distribution over the number of topics you let the tool made. In the context of this study, it meant that theoretically, fifty topics were present in each song, and the sum of their contributions was always one. However, for each song there was of course a vast majority of topics that was not present, and their contribution was very close to (but never) zero. At the same time there was always one topic that stood out as most dominant. In a lot of cases, the contribution of that topic was quite high (0.6 or higher), but there were also songs in which the topic with the highest contribution still had a very low contribution, in the worst cases even below 0.2. To be pragmatic I have limited my choice to the most dominant topic for each song (regardless how small or large the contribution of that topic), but I realize that this undoubtedly has biased my results.

In answering the second part of the research question -- investigating topical fluctuations in a diachronic corpus -- the curve of the horizon Underwood spoke about, came into view. I zoomed out in order to see which patterns became visible by observing topics over a longer period of time. What I learned is that, speaking in general, every topic peaked at one particular moment. It means that most of the topics are not equally dominant at all times, but appear, peak and disappear at a certain moment. To what extent is this relatable to spelling and word choice? Is it possible that songs from the same period stick together anyway, because they look like each other regarding their words, and the way these words are spelled? This is exactly what I have tried to eliminate by using a spelling normalization tool, but since this tool has its limits as well, it is possible that I only partly succeeded in this. Still, intuitively I can not imagine that this is the only reason for the sharp peak in every topical plot. Above all, it suggests how certain topics appear and disappear over time, making room for other topics which are often very similar, but still at an important point different, otherwise these songs would have had the same dominant topic.

In my thesis, I have related topical fluctuations to cultural-historical changes. It turned out that several topics can be related to certain events in the Dutch Republic, for example the rise of Petrarchism, the Eighty Years' War and the heydays of the Further Reformation. A question that rises from these results is: are songs a predictor or follower of trends? Do songs arise as the result of a certain trend, or are they boosting a certain trend? From previous, qualitative studies, we know that both options were common in the early modern times. The number of songs of the Further Reformation, for example, grew because Calvinists published their own songbooks in high circulation. In these songs, subjects were described which were already discussed and preached about in churches. The Wilhelmus, however, is an example of a song that functioned more as a booster of a trend, namely the revolt against the Spanish king. In the end, I think that the current results are too roughly meshed to say something wise about this matter. Other methods will undoubtedly be more suitable to answer this question.

More than exploring the relationship between topical fluctuations in my results on the one hand and cultural-historical changes in the Dutch Republic on the other hand, I mainly explored how my results, obtained with quantitative research, can be related to the \textit{status quaestionis} in qualitative research on Dutch early modern literature. Which roles are assigned to literature, concerning the early modern centuries? I distinguished four of them and explored whether my obtained results could validate these four theses. Regarding the relationship between literature and ideology, I illustrated with songs from the Further Reformation how a certain ideology is carried out in song lyrics. The relationship between literature and poetics of which we know, thanks to numerous qualitative studies on humanism, the classical tradition, the Renaissance tradition, and Petrarchism, was confirmed by the quantitative results regarding love topics in my corpus. Concerning the third thesis on early modern literature -- literature and politics are closely related in times of political crises -- this was confirmed mainly by the enormous number of songs with \textit{nation and country} as their most dominant topic, published during the Revolt. The last claim on the role of literature in the early modern times -- diversity in both involved people and topics -- was more complicated to test. With the used method, it is not possible to distill an intended public from a text. I already stated that other methods would be more fruitful to test this thesis. The diversity in topics I have investigated in calculating the turnover.

In calculating the turnover of topics, I came close to the theoretical field I wanted to explore in this thesis: cultural evolution. Hearing and reading about it in the months before I started to write my thesis, I had become curious about \enquote{how Darwinian theory can explain human culture and synthesize the social sciences}, to say it with Mesoudi.\autocite{mesoudi_cultural_2011} I wanted to investigate if and how cultural evolution could be applied to early modern songs. But when does a study become a study on cultural evolution? It is more than offering a diachronic perspective. Cultural evolution links certain changes in texts, or any cultural product, to cultural and historical circumstances, and human behavior. I did the \enquote{Darwin check} to see if the three preconditions of biological evolution -- variation, competition and inheritance -- could be applied to culture, and more specific, to a corpus of Dutch early modern songs. The songs survived the test, and I explored how I would be able to say something about the transmission bias that might have been at stake in human behavior when inheriting song topics. In the end, I did not gain the unambiguous result I was hoping for. I faced the limitations of either my corpus, my skills and knowledge, or the available methods.

I am almost tempted to say that culture -- and in this case, a complex corpus of songs, uneven distributed over time, and with loads of spelling variation -- is too complex for simple evolutionary models. But then Mesoudi refutes:

\begin{quote}
	It is not being argued that human culture really can be described entirely in terms of just a handful of simple variables and processes. Rather, the reduction involved is methodological: faced with an enormously complex phenomenon such as culture, the most productive approach to understanding that phenomenon is to divide it up in small pieces and try to explain them in a piecemeal style. (...) Only simple models can yield quantitative, testable predictions that can be shown to be either statistically supported or not; and if not, we simply build and test an alternative model.\autocite[131]{mesoudi_cultural_2011}
\end{quote}

\noindent Against another objection, namely that cultural phenomena are too unique and particular to be compared with each other in the way cultural evolutionists suggest, Mesoudi argues with stating that

\begin{quote}
	the aim is not to \textit{replace} the study of specific historical events or societies with general microevolutionary processes. On the contrary, without details of specific societies or time periods, there is no data with which to test the validity of the general evolutionary processes. Nor is it to deny that there are important differences between societies and time periods. Of course there are. But there are also similarities, and identifying these similarities, and explaining them with simple principles (...) surely \textit{adds} to the study of specific cases rather than detracts from them.\autocite[132]{mesoudi_cultural_2011}
\end{quote}

\noindent It is quite funny to recognize the same rhetoric as used by digital humanities scholars such as Underwood and Piper, in defending their computational methods, just as I did in the beginning of this chapter, as well as in Chapter 2. I might not have understood yet the width and depth of cultural evolution by writing this one thesis, what I do know is this: to investigate whether we can indicate cultural evolution in a corpus of Dutch early modern songs, quantitative research of these songs is necessary. But, at the same time, in order to understand the observed patterns achieved with computational methods, we need the results obtained with qualitatively researching early modern literature as well.

Using computational methods on large digital corpora, is like zooming out on Google Earth: you do not see the streets, the squares and the roundabouts anymore. Instead, you start to see cities, borders, huge woods, long rivers, mountains, deserts, oceans, and, eventually, the shape of the earth. It changes your whole perspective on the city you reside in, the country you are inhabitant of, and the continent on which you live. But to plan your route to visit a concert of Famke Louise on a pop stage you have not been before, you still use Google Maps. And that is okay. Because the one does not have to replace the other.


