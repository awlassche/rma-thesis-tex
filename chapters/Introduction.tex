\chapter{Introduction}

\epigraph{\textit{Ghy loopt by al die Vrijsters\\
	En maeckt veel waters vuyl:\\
	Met suycker soete woorden\\
	Payt ghy den gantschen hoop;\\
	Het schijnen mooye Boonen,\\
	Doch woorden zijn goet koop}}{--- Anonymous (1645)}

\epigraph{\textit{Waarom denk je dat je kans maakt boy,\\
		Denk niet dat het zo is,\\
		Al je doekoe in een Louis store, je bankstatus is niks}}{--- Famke Louise, \textit{Op Me Monnie} (2017)}

\noindent In a recent video on the Dutch YouTube channel \textit{VoxPop}, Roel Maalderink asks middle-aged people on the streets what they think of the genre of Dutch hip hop (\enquote{Nederhop}). When they collectively express their disdain for these songs, Maalderink lets them read some lyrics of what he tells them are \enquote{eighteenth century songs}, which are in fact just archaic translations of Dutch hip hop lyrics. Among these is this one, by the \enquote{eighteenth century poet Ali Batavi}:

\begin{quote}
	Want dit is\\
	die leipe Batavenodeur\\
	Geen probleem Philips II \\
	brengt de leipe Batavenodeur
\end{quote}

\noindent Every not-so-middle-aged listener will immediately recognize the hip hop song \enquote{Leipe Mocro Flavour} (2004) by Ali B, Yes-R and Brace, but the middle-aged ladies and gentlemen in the video have no clue, and are impressed by the poetical ingenuity of Ali Batavi and other songwriters, including \enquote{De jongelui van heden ten dage} and \enquote{Boëf}. Although Maalderink does not realize that Philips the Second was a sixteenth-century king who had nothing to do with the eighteenth century Batavian Republic, the outcome of the video is clear: as soon as contemporary hip hop lyrics are translated into historical Dutch, baby boomers see the beauty of it as well.

This provisional experiment raises a number of questions. To what extent is the idea true that topics and themes in song lyrics have changed over time, if it takes just a quick translation to let people over fifty enjoy the same lyrics the younger generation does? How do contemporary songs differ from historical songs, except for their spelling and word choice? In terms of theme, the quoted anonymous song from 1645 seems quite similar to Famke Louise's \enquote{Op Me Monnie}. What did people sing about during the early modern times? What changes can be observed over a long period of time? If there is one thing we know about the early modern Dutch song culture, it is the fact that it was an excelling field. According to Louis Peter Grijp, song culture is as typically Dutch as the culture of painting: \enquote{men is groot in het kleine} \autocite[29]{grijp_het_1991}. Furthermore, Natascha Veldhorst argues that Dutch song culture \enquote{held the entire population in thrall, from young to old, from poor to rich} \autocite[217]{veldhorst_pharmacy_2008}. For decades, literary scholars on the early modern period have published studies on historical songs and song culture, which contrasts with the small number of studies on contemporary Dutch song culture. Only since a few years, this has become object of scrutiny for literary scholars, resulting in several studies, including ones on Dutch pop song lyrics, and representation in Dutch hip hop.\footnote{See for example \autocites{buelens_listen_2013}{roest_buurtvaders._2017}.} In sharp contrast to the large number of case studies on historical Dutch song culture\footnote{See for example \autocites{stronks_dees_2014}{stronks_stichten_1996-1}.} stands a very small number of \textit{quantitative} studies on Dutch early modern songs. In fact, Grijp is the only one who undertook a successful attempt to mapping the Dutch early modern song culture, by writing his dissertation \textit{Het Nederlandse lied in de Gouden Eeuw. Het mechanisme van de contrafactuur} (1991). In this work, Grijp reconstructs the melodies of the songs, by finding their sources and in determining their reception. One of the results of his extensive work is the birth of the so-called \enquote{voetenbank}, a primitive electronic database containing information on melodies, verses and stanzas of (at that time) 5,700 Dutch historical songs \autocite[13]{grijp_het_1991}.

In her work on Dutch early modern songbooks, Veldhorst states that we do not know enough about the early modern song production, to come up with specific numbers. In the first place we do not have an overview of all early modern songbooks that are kept in European libraries nowadays. Secondly, we need to keep in mind that songbooks were heavily modified, expanded, and revised in the course of time. Thirdly, the small survival chances of the songbook further complicate any attempt at estimating \autocite[225-227]{veldhorst_pharmacy_2008}. This supposed lack of data accounts for the relative scarcity of quantitative studies on early modern songs. If there is one thing you need in order to perform quantitative research, it is numbers and data. However, almost thirty years after the publication of Grijp's dissertation, the \enquote{voetenbank} has grown into the extensive Dutch Song Database (DSD, \textit{Nederlandse Liederenbank}, hosted at the KNAW Meertens Institute), containing metadata of more than 175,000 Dutch songs. Moreover, in the last couple of decades the opportunities to explore a corpus digitally have grown exponentially. While Grijp had to manually count and search for patterns, we have, thanks to the rapidly growing field of the digital humanities, the computer to perform these kind of tasks. Computational tools allow us to research a massive corpus and to count and investigate characteristics in ways not previously available. This thesis uses these new possibilities to quantitatively research Dutch early modern songs and explore how the outcome of quantitative research relates to qualitative research done so far on early modern songs, and, in a broader perspective, on early modern literature.

In doing so, I build on earlier research on the same corpus, in which we examined what kind of effect repetition in song lyrics has on the popularity of a song. Previous studies on repetition and popularity of pop songs showed that repetitive lyrics have a positive effect on the popularity of a song. \autocites{nunes_power_2015}{alexander_entropy_1996} In our study, we confirmed that a similar pattern can be found regarding early modern Dutch songs. After undertaking a series of \enquote{Survival Analyses}, we analyzed the use of word repetition as a determinant for the survival patterns of religious and profane songs. Results showed a significant effect of word entropy on the survival chances of a song, where songs with more repeated words (i.e. lower entropy values) had longer lifespans and higher reprint probability.\autocite{lassche_repetition_2019} As a sequel to this earlier study on a song's form, I will examine the content of a song in this thesis. In order to establish which topics were dominant in the lyrics of early modern Dutch songs, I will perform topic modelling on my corpus. I assume that if a topic becomes more prevalent, the words describing it become more frequent as well \autocite[6]{karjus_quantifying_2018}. I will use MALLET's adaptation of latent Dirichlet allocation (LDA) in the Python package \texttt{gensim}, in order to build topics from my corpus.

I will analyze which topics were more dominant than others, which dynamics can be observed in the fluctuations of topics in my corpus over time, and how changes in these topic distributions can be related to cultural and historical real-world changes. For this last step I draw inspiration from the field of cultural evolution. This research area provides a series of concepts that allow us to understand what drives cultural trends. Cultural evolution draws a parallel between genetic evolution and cultural change, arguing that changes in socially transmitted beliefs, knowledge, languages, and so on, can be described in terms of just the three basic preconditions with which all biological change is described by Charles Darwin in \textit{The Origin of Species} (1859): variation, competition and inheritance \autocite[26]{mesoudi_cultural_2011}. Moreover, following Andres Karjus et al., I assume that \enquote{contexts, or topics, tend to change with the times, along with the world that they describe} and that \enquote{[t]hese changes are expected to be reflected in (...) diachronic corpora} \autocite[1]{karjus_quantifying_2018}. In analyzing fluctuations in topic distributions, I will test how theses from contemporary qualitative research about the role of literature in the early modern centuries, can be related to my quantitative results. Wrapping this all up in one sentence, the two research questions I aim to answer in this thesis are the following:

\begin{quote}
	\textit{What are the most popular topics in Dutch early modern songs and how do topical fluctuations in a diachronic corpus of Dutch songs relate to cultural-historical changes and contemporary theses in qualitative research on the role of literature in early modern centuries?}
\end{quote} 

\subsubsection{Structure of this thesis}
To answer my research question, I use the following structure. In the remainder of this first chapter I will give a brief introduction to the cultural-historical field of Dutch early modern song culture. Furthermore, I will discuss in more detail some of the prevailing theses in contemporary qualitative research on the role of literature in the early modern times. The rest of this thesis I have divided in five parts. Part I, \textit{Theoretical framework}, contains two chapters in which the two fields in which this research is embedded are discussed. Chapter 2 concerns the field of digital humanities. After a broad introduction, I will zoom in on the debate on computational literary studies, and discuss existing methods to extract themes from a corpus. In Chapter 3, I will examine the other theoretical pillar of my research: the field of cultural evolution. I will explore how the concepts of variation, competition and inheritance can be applied to culture. Subsequently I introduce the terms \enquote{transmission bias} and \enquote{turnover}. The next part, \textit{Methodology}, consists of three chapters. In Chapter 4 I will introduce my corpus and discuss the main corpus characteristics. The next chapter is about two methods I tested to perform spelling normalization on my corpus. Chapter 6 contains an in-depth discussion of Latent Dirichlet Allocation (LDA), the topic modeling method I used. This chapter also includes a discussion of the  methodology concerning the calculation of turnover series. Chapters 7 and 8 belong to part III of this thesis, the \textit{Analysis and results}. In Chapter 7 I report the steps that I have taken in building topic models from my corpus, which are critically evaluated afterwards. In Chapter 8 I investigate how the observed topical fluctuations relate to four prevailing claims, drawn from qualitative research, on the role of literature in early modern times. Part IV, \textit{Conclusion} contains a chapter in which on the one hand the implications of these interpreted results, and on the other hand future work will be discussed. Part V, the final part of this thesis, contains all appendices.

\subsubsection{Early modern literature and song culture}
At the beginning of the seventeenth century, Dutch literature flourished like never before. Karel Porteman and Mieke B. Smits-Veldt summarize it well in the title of their contribution to the series \enquote{Geschiedenis van de Nederlandse literatuur}: \textit{Een nieuw vaderland voor de muzen}, in other words, a new homeland for the muses. On the first pages of their extensive study, Porteman and Smits-Veldt explain how the words \enquote{homeland} and \enquote{muses} indicate a new phase in the history of Dutch literature. The muses refer to the \enquote{arts}, in particular to poetry, an art form that, after a long journey from the mountains Helicon and Parnassus, then successively via the Romans, Italy and France, finally had arrived in the Low Countries in the second half of the sixteenth century. It was the birth of Dutch renaissance literature.\autocite[17]{porteman_een_2009} The term \enquote{homeland} refers to the idea that poets, in writing their poetry, contributed to the idea of a shared homeland with a shared language. The muses therefore not only came to live in a new homeland, but also gave it shape.\autocite[18]{porteman_een_2009}

About the role of literature in the early modern times, a few claims are prevailing in current qualitative studies. I determine four different theses, which all have the idea in common that literature plays a major role in the construction of an identity. This did not only happen by poets contributing to the idea of a shared homeland and language during a long lasting war for political and religious independence, but also by a growing focus on the contradiction between the local and the global. During the seventeenth century, Porteman and Smits-Veldt observe a regionalization of literature. Although Amsterdam remained the cultural center of the Dutch Republic, new poets from other places in the provinces Zeeland, Holland, and Friesland made themselves heard.\autocite[21]{porteman_een_2009} 

The first claim is that literature functioned as propagator of an ideology. This happened for example in the literature belonging to the pietistic tradition which was rising in the second half of the seventeenth century. In the Dutch historiography, this movement is sometimes referred to as the Further Reformation. Pietism appealed to the pre-Reformational piety, in which the inner faith experience was centralized.\autocite{eijnatten_nederlandse_2006} A key concept of pietism was the idea of \enquote{bevindelijkheid}, meaning that protestant belief should not only be manifested in all aspects of daily life, but especially in an inner experienced reality. According to Porteman and Smits-Veldt, two trends can be determined in the literature of the Further Reformation, which started around 1650. On the one hand, there were literary texts in which an external program was realized, focusing on a sanctification of society and all aspects of life: family, school, church, and public morals. On the other hand, texts appeared in which the personal lived experience with God was described and expressed. One of the literary text genres through which the dissemination of these ideas took place, were songs, since free songs (as in, songs that are not psalms) were no longer seen as problematic in Calvinistic environments. Proponents of the Further Reformation published their own catechisms, sermon bundles, and songbooks, causing a quick rise of the number of reformed songbooks after 1650.\autocite[658-660]{porteman_een_2009} This is also what Els Stronks showed in her dissertation \textit{Stichten of schitteren. De poëzie van zeventiende-eeuwse gereformeerde predikanten}. She explores how through the means of cheaply produced song collections and very pious and instructive song texts, attempts were made to raise and purify the lifestyle and behavior of the Calvinists.\autocite[36-49]{stronks_stichten_1996-1} Besides, Stronks demonstrated how some of these poets also hoped for a small literary career.

The relationship between literature and poetics is the second claim from qualitative research I want to investigate. With \enquote{poetics}, I mean the idea that poetry of early modern Dutch poets became more and more embedded in a Renaissance and classical tradition. Genres and topics from these traditions were appropriated by Dutch poets. One characteristic was the focus on and appreciation for the vernacular. Glorifying the native language was accompanied by the purification and expansion of language.\autocite[40]{porteman_een_2009} How poets from countries such as Italy and France adapted the Greek and Latin tradition, should also be able in the Dutch native language, was the thought. An important example of this embedding in the Renaissance tradition is the practice of the Petrarchistic genre. Dutch poets wanted to demonstrate that they too belonged to the refined European culture which celebration of love owed much to the Italian Renaissance poet Petrarch.\autocite[131]{gelderblom_investing_2007} Other examples are the roles that were occupied by mythological figures in literature. With the move of the \enquote{muses} to the Low Countries, all these aspects became more prevalent in Dutch literature.

Another starting point that is often used in contemporary literary research, is the existence of a relation between literature and politics. Examples of this relation are studies on Joost van den Vondel's politically charged play \textit{Palamedes oft vermoorde onnoozelheit} (1625), in which Vondel allegorically used the classical Greek hero Palamedes to convert the event of the execution of Johan van Oldenbarnevelt in 1619 to a martyr's story with classical allure. In the article \enquote{Lezers in de marges van Vondels \textit{Palamedes}} for example, we explored how the \textit{Palamedes} was read in the seventeenth century. We found that, given the sales numbers of the play, and the nature of the marginalia in the survived copies, the play experienced renewed popularity during the First Stadtholderless period (1650 - 1672), culminating in the year 1672 (the \enquote{Rampjaar}) in which the murder of statesmen Johan and Cornelis de Witt took place. \autocite[36, 50]{lassche_lezers_2019-2} This suggests that, in times of political crises, literature with a political connotation becomes extra important: it is read more often. Moreover, literature with a political undertone is written more often during political crises, and can shape and intensify the debate. Roeland Harms shows this in his dissertation on public opinion titled \textit{Pamfletten en publieke opinie. Massamedia in de zeventiende eeuw}. In this, he demonstrates how the production of political pamphlets enabled the reader to be extensively informed about the viewpoints of different political parties, which eventually led to the public at large participating in the political debate, and hence to the invention of public opinion.\autocite[256]{harms_pamfletten_2011}

Returning to the text genre that is key in this thesis -- the song -- the most striking and well-known example of the strong connection between literature and politics is undoubtedly the song that became the Dutch national anthem in 1932: the Wilhelmus. This song, written in the late sixteenth century and containing no less than sixteen verses, voices William of Orange. The Wilhelmus functioned as a protest song, by which William of Orange gathered supporters to revolt against the Spanish rule, which resulted in the Eighty Year's War. It is not surprising that William of Orange choose a song to gather his crew: during the early modern period, the tradition of the monodic\footnote{Unisono.} song in the vernacular was extremely popular in the Dutch Republic. Tens of thousands of Dutch songs were written, published, read and, obviously, sung. According to Veldhorst, everyone sang everywhere and everytime: singing was a second nature of the Dutch \autocite[12]{veldhorst_zingend_2009}. Proof of that can not only be found on contemporary paintings, but also in archives and libraries, where hundreds of early modern Dutch songbooks are still kept. The singing tradition was especially flowering in the seventeenth century: it rained songbooks at that time, Gerrit Kalff once stated.\autocite[666]{kalff_het_1884} The above made claim, how during the early modern centuries larger and more varied groups of people got involved in literature, is the fourth that is often used in contemporary research and I want to test in this thesis. Not only the listeners and writers of songs became more diverse: the different number of occasions and subjects on which songs were made, expanded as well. Veldhorst determines laments, wedding songs, prayers, lovers' dialogues, shepherds' plaints and panegyrics. \autocite[224]{veldhorst_pharmacy_2008}

Most of the songs written in the early modern centuries were \textit{contrafacta}, which means that a new text is written on an existing melody. One of the reasons for writing \textit{contrafacta} rather than composing new songs (with both a new text and a new melody) was, according to Grijp, the combination of a weak musical and a strong literary culture \autocite[23, 24]{grijp_het_1991}. Imagine you have to write a song for a wedding. It is much easier to write a text on an existing melody, than to compose a new melody and write a new text to it. And even if you might be a musical genius and are able to write a song with both a new text and a new melody, hardly anyone would be able to sing along, since they do not recognize the melody. The opportunity to sing along was therefore another importing consideration to choose in these kind of situations for a contrafacta.

The demarcation of the term \enquote{songbook} is not as unambiguous as one might expect. There is a wide variation between the different books within the genre, for example regarding the format, content and the presence or absence of musical notation or melody name. I use Veldhorst's definition that a songbook is a printed collection of songs in the vernacular, sometimes mingled with other poetry, and most often (but not always) with a melody assigning above the song, instead of musical notation. \autocite[15]{veldhorst_zingend_2009} The content of a songbook was not static: there was a lively exchange of songs between the producers of songbooks. Publishers were accustomed to copying songs from earlier songbooks or books printed elsewhere. Sometimes this even resulted in fifteen or more appearances of one popular song in different songbooks. \autocite[73-74]{veldhorst_zingend_2009}

In this study I use a sample of the Dutch Song Database, which is in principle not designed to be a repository of songbooks, but a database of individual songs. However, this database also contains metadata about the songbooks a song appeared in. I will focus on Dutch songs from 1550-1750. A motivation for this specific corpus definition, as well as a more in-depth explanation of the characteristics of the corpus, will be given in Chapter 4, \textit{Corpus}. With the methods used in this thesis, I will validate or evaluate the above mentioned claims, drawn from qualitative research, on the role of literature in Chapter 8, \textit{Roles of literature}.