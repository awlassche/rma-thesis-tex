\chapter{Cultural evolution}
To analyze the dynamical changes in the most dominant topics in early modern songs, I draw inspiration from the field of cultural evolution. In this chapter I will first explain how this field originates from Darwin's theory of biological evolution. I continue by investigating if and how the three preconditions of evolution can be applied to culture. This chapter closes with a section on biases and patterns that can be observed in human copying behavior.

\section{From biological to cultural evolution}
Before I will explain in more detail what cultural evolution comprises, we need to go 160 years back in time, to the moment in 1859 in which \textit{The Origin of Species} of Charles Darwin saw the light of day. For the first time, \enquote{somebody had set out (...) a tenable scientific explanation for two phenomena that had predominantly been attributed to supernatural, mystical, or religious forces}, as Alex Mesoudi puts it.\autocite[vii]{mesoudi_cultural_2011} The two phenomena he is referring to, are \textit{(i)} the diversity of biological organisms and \textit{(ii)} the complexity of their adaptations. Darwin argued that these could be explained by three simple principles: variation, competition and inheritance.\autocite{darwin_origin_1979} The consequence is natural selection, according to Darwin: the characteristics that increase the chance of an individual to survive and reproduce, are more likely to be inherited by the next generation. In the end, such characteristics will increase in frequency within a population.\autocite[viii]{mesoudi_cultural_2011} Although Darwin himself already drew parallels between biological evolution and cultural change, more specific the change of language, it is only since a few decades that researchers in the social sciences and humanities have showed that the principles that Darwin applied to biological change a century and a half ago, can also be applied to culture. However, this does not mean that scholars such as historians, literary researchers and linguists have not investigated and described large-scale patterns of cultural change before. What is new though, is the use of rigorous, quantitative methods that are originally designed for explaining patterns and trends in biological evolution.

To understand in more detail how cultural evolution works, let me first define culture. I borrow Mesoudi's definition, postulated in his book \textit{Cultural Evolution. How Darwinian Theory Can Explain Human Culture and Synthesize the Social Sciences}:

\begin{quote}
	[C]ulture is information that is acquired from other individuals via social transmission mechanisms such as imitation, teaching, or language.\autocite[2-3]{mesoudi_cultural_2011}
\end{quote}

\noindent This definition ties in with what Edward Burnett Tylor defined as culture in 1871, namely \enquote{that complex whole which includes knowledge, belief, art, law, morals, custom, and any other capabilities and habits acquired by man as a member of society}\autocite{tylor_primitive_1871}, with the important difference that Mesoudi does not restrict culture to the male half of species. His broad definition of culture also includes literature, music, and art and therefore as well the subject of this research: songs. In this definition, culture is defined as information rather than behavior, but that does not mean that \enquote{culturally acquired information does not \textit{affect} behavior.}\autocite[3]{mesoudi_cultural_2011} This distinction is important to emphasize, because if behavior is the thing we are trying to explain, then including it in the definition of culture makes cultural explanations for behavior circular: a thing cannot explain itself. Furthermore, there are other causes of behavior besides culture.\autocite[4]{mesoudi_cultural_2011}

Mesoudi signals problems with the way culture is studied in the last decades. He quotes evolutionary psychologists John Tooby and Leda Cosmides who have lamented that, during the second part of the twentieth century, every aspect of human behavior was explained in terms of some mysterious force called \enquote{culture}. Mesoudi considers this complaint a just one: in his opinion, \enquote{social scientists studying cultural phenomena have been (...) unable to specify in precise terms exactly how culture operates beyond some vague and informal notion of \enquote*{socialization} or \enquote*{social influence}.}\autocite[18]{mesoudi_cultural_2011} There is not only a lack of quantitative, formal methods and scientific hypothesis testing, but culture is also treated as static rather than dynamic, and there is a general lack of communication of findings and methods across different branches of the social sciences. This is where the theory of cultural evolution comes in: it can provide solutions to the problems mentioned above. First, it fully recognizes the role of culture in explanations of human behavior. Furthermore, it provides formal, quantitative methods that can be used to explain cultural phenomena in a way that explicitly incorporates change over time.\autocite[22-23]{mesoudi_cultural_2011}

\section{Variation, competition and inheritance}
Although biologists have, since the publication of \textit{The Origin of Species}, established that Darwin's theory about biological change is correct, the history of the concept of cultural evolution is still controversial. Darwin's idea was that, if the three basic preconditions -- variation, competition and inheritance -- of natural selection can be demonstrated, evolution happens. At the same time, if the three preconditions can not be demonstrated, evolution does not happen. To verify if cultural evolution is a valid approach, the question is therefore if the three Darwinian preconditions can be applied to culture.

The first precondition Darwin distinguishes, is \textit{variation} between individuals of several species, in his case mostly pigeons. Applied to culture, one can say that people vary in their religious beliefs, political views, skills, hobby's and so on. These variations in mental aspects of culture, results in variation in tools. The next step is to see if we can give documented examples of this variation and verify if we can quantify it. Mesoudi illustrates this with the example of historian Henry Petroski, who documented variation in forks founded in the late 1800s. All sorts of forks he found varied in the number of tines, the dimensions, the shape of the tines, the material used, and so on.\autocite[28]{mesoudi_cultural_2011} Applied to the subject of this thesis, one can say that songs also show variation. Genres as love songs and religious songs can be distinguished, and within a genre there is often a further variation: we can list love songs that focus on physical attraction, on love as a gift from God, on the impossible love, and so on. Since we have a database of these songs, this variation can be quantified. It is therefore reasonable to conclude, following Mesoudi, that variation is present in human culture, and that this variation can be documented and quantified. We can state that the first of Darwin's preconditions is present in culture.\autocite[29]{mesoudi_cultural_2011}

Thinking of \textit{competition}, the second precondition of Darwinian evolution, one might think of a situation in which two individuals are fighting and the biggest or strongest or smartest individual wins. Competition can be indeed physical, but it does not have to be that direct. An example of indirect competition can be two plants of which one is more able to deal with dry conditions than the other, and therefore has a higher chance to survive in a dry season than the other.\autocite[30]{mesoudi_cultural_2011} It is obvious that in culture some sort of competition is present as well. Darwin already gave an example of competition in language himself:

\begin{quote}
	A struggle for life is constantly going on amongst the words and grammatical forms in each language. The better, the shorter, the easier forms are constantly gaining the upper hand.\autocite[91]{darwin_descent_2003}
\end{quote}

\noindent Turning again to the subject of this research, this \enquote{struggle for life} can also be adapted to songs. Taking two songs of which one turns out to be the hit of a decade, while the other is forgotten after a couple of weeks, they function as examples of competition as well. The former is replacing the latter, as a result of competition between the two songs. My previous research\autocite{lassche_repetition_2019}, in which we concluded that songs with more repeated words have a longer lifespan, can also be seen in this light of Darwinian competition. We can conclude with Mesoudi that cultural traits, like biological organisms, take part in an endless struggle for existence.\autocite[31]{mesoudi_cultural_2011}

The third and final precondition for evolution that Darwin distinguished, is \textit{inheritance}. In a biological context, this means that individuals that are more likely to survive and reproduce during the struggle for existence, will often pass their traits or characteristics to their offspring. We can assume that these traits are, at least partially, responsible for the parents' increased chances of survival and reproduction. The result is that there will be a gradual increase in fitness and adaptation to the local environment. Therefore, without inheritance beneficial traits are not preserved, and evolution cannot occur.\autocite[32]{mesoudi_cultural_2011} The question is whether cultural information can be successfully reproduced, or transmitted, from one person to another as well.\autocite[32]{mesoudi_cultural_2011} Cross-cultural studies have shown that people acquire beliefs, attitudes, skills and knowledge from other people via cultural transmission. On a smaller scale, cultural transmission can also take place from one person to another. The transmission from adult to child is the most common example. However, there is more to say about inheritance. Darwin also was convinced that \enquote{for biological evolution to work, minor modifications must not merely be inherited from parent to offspring in a one-to-one fashion, inheritance must be faithful enough such that potentially combined with other beneficial traits.}\autocite[33]{mesoudi_cultural_2011} Mesoudi explains that, in culture, we can see this gradual accumulation of modifications as well. He illustrates this with the example of successful innovations, such as the steam engine, which rarely, if ever, spring out of nothing. Rather than suddenly emerging fully formed from a genius' mind, innovations are often a modified version of a preexisting artifact.\autocite[33]{mesoudi_cultural_2011} The same can be said about a specific song, of which the words or melody can be slightly modified through time.

Now that we have seen that human culture also exhibits the last of Darwin's three preconditions for evolution, one can conclude that Darwinian evolution is a valid approach to investigate and understand culture. However, cultural evolution does not take place in a purely random way, but there are cognitive mechanisms that guide the evolution. In the next paragraph I will elaborate on the fact that humans use transmission biases to choose when, what or from whom, to copy.

\section{Transmission biases and turnover series}
Drawing the parallel between biological evolution and cultural evolution even further, \enquote{natural selection} becomes \enquote{cultural selection}. Mesoudi defines this as the process that occurs when \enquote{one cultural trait is more likely to be acquired than an alternative trait.}\autocite[56]{mesoudi_cultural_2011} Several kinds of learning biases that facilitate this selection of useful traits can be distinguished. Different terms have been assigned to these biases, but in general two learning mechanisms are determined: \textit{content-based} biases and \textit{context-based} biases.\autocites[56]{mesoudi_cultural_2011}[129]{henrich_evolution_2003,boyd_culture_1985} The context-based biases can be further split up into \textit{model-based }biases and \textit{frequency-dependent} biases, as Mesoudi calls them.\autocite[56]{mesoudi_cultural_2011} Furthermore, according to Alberto Acerbi and Alexander R. Bentley, the frequency-dependent bias can either be a \textit{conformity} bias or an \textit{anti-conformity} bias.\autocite[229]{acerbi_biases_2014} Content-based biases affect the likelihood of a particular cultural trait being transmitted because of the intrinsic attractiveness of this trait. In other words: the inherent features of a trait determine the choice for it.\autocites[65]{mesoudi_cultural_2011}[129]{ henrich_evolution_2003} For example, Acerbi and Bentley mention a study of Olivier Morin who explained the success of direct-gaze portraits over indirect-gaze ones with a content-based bias, namely that a direct eye-gaze is more cognitive appealing.\autocites{morin_how_2013}[as cited in][228]{acerbi_biases_2014} The higher reprint probability of early modern Dutch songs with lower entropy values (i.e. more repetition in their lyrics) can also be explained as a content-based bias.\autocite{lassche_repetition_2019}

A context-based bias arises from the context of a cultural trait. The choice is not directly determined by the features of the traits, but by external determinants. Context-based biases are often subdivided in model-based biases and frequency-dependent biases. A model-based bias occurs when people preferentially adopt certain cultural traits on the basis of the characteristics of the model exhibiting them. Examples can be a prestige-based bias (when people prefer to copy prestigious models who have high social status) or a similarity-based bias (when people prefer to copy models who are similar to them in dress, beliefs, dialect).\autocites[73]{mesoudi_cultural_2011}[129]{henrich_evolution_2003} When engaging in the frequency-dependent bias, people use the frequency of a treat in a population as a guide as to whether adopt it, irrespective of its intrinsic characteristics.\autocite[71]{mesoudi_cultural_2011} This results in two kinds of frequency-dependent biases. The first is the one where popular traits are preferred in respect to the unpopular ones: this is the conformity bias. The opposite is also possible, where unpopular traits are preferred in respect to the popular ones. Here transmission is negatively frequency-dependent: this is the non-conformity bias.\autocite[229]{acerbi_biases_2014} A neutral model, which assumes that copying is completely random and unbiased, would produce characteristic right-bended distributions, where very few traits are very popular, and the vast majority of traits remain rare. Adding a conformity bias to the neutral model results in a distribution which is even more bended than the one produced by a neutral model, since popular traits are proportionally more advantaged. The anti-conformity bias produces instead distributions where the majority of traits result at intermediate frequencies.\autocite[229]{acerbi_biases_2014}

Which biases are at stake in the cultural evolution of historical songs? To detect the transmission biases in the cultural evolution of early modern song topics, I will study the \enquote{turnover series} of these topics, in order to quantitatively map cultural changes in it. The turnover can be defined as the number of new traits that enter a top-list of a certain size for a certain cultural domain. Popular examples of turnover calculations are the number of new entries in hit charts, but it is possible to calculate turnovers for any cultural domain. Acerbi and Bentley showed that the turnover of recent baby names in the United States produces a \enquote{concave} turnover profile (decreasing slope), signalling that people tend to prefer relatively uncommon names and thus that there is an anti-conformist bias. This contrasts with the case of early boys names, where a \enquote{convex} turnover profile (increasing slope) can be distinguished.\autocite{acerbi_biases_2014} Other examples are studies on the frequency of pottery designs, word usage or bird song elements.\autocites{bentley_random_2004,bentley_random_2008,byers_independent_2010}[as cited in][229]{acerbi_biases_2014} In Chapter 6, I will further explain how I am going to use and calculate turnover series in the context of this study.